\chapter{Structure}

In the previous chapter we investigated the dynamics of three molecular systems; \sone, \scon and \tri. The dynamics properties of these systems remained smooth throughout the range we simulated them, indicating that there was no phase transition taking place. While there is no obvious phase transition, one possible cause of the increasing time scales for the dynamics properties is the growth of crystal regions within the simulations.

The goal of this chapter is to investigate the structural properties of the three systems studied in the previous chapter; the \sone, \scon and \tri to determine whether the dynamic properties observed are purely dynamic properties or can be attributed to structural properties of the system.

\section{Is there Crystal Formation}

Over the course of the simulations to determine the dynamic properties of the system it is possible that the system transitioned from the purely amorphous phase to a structure that was a mixture of crystalline and amorphous regions. If these regions are long lived, rather than just fluctuations, as the temperature is dropped these regions are only more likely to grow; the lower entropy crystal phase is more favourable at low temperatures. Using this reasoning we can use the final configuration as an indication of the formation of crystalline regions over the entire temperature range of the simulation. From this final configuration we can take an inherent structure, removing any of the vibrational noise giving a configuration that is representative of the mean arrangement of the molecules. These arrangements are shown in \textfigref{inherent structures frame} where the colour of each molecule is given by its orientation.
\towrite{discuss inherent structures}

\begin{figure}
    \todofigure{Inherent structures}
    \caption{Inherent structures}
    \label{fig:inherent structures frame}
\end{figure}

While there appears to be no long range ordering in these configurations, to make a definitive statement we need more specialised tools. One of the alternate forms that could form is a plastic-crystal, a form where the position of the molecules is ordered while their orientations are disordered. This form can be more easily discerned by removing the orientational representation of the molecules, instead representing the center of mass of each molecule as a dot~\figref{inherent structures com}. This focuses the search for order, the centers of mass lining up over long ranges would be indicative of this plasti-crystal order.

\begin{figure}
    \todofigure{Inherent structure COM}
    \caption{COM of inherent structures}
    \label{fig:inherent structures com}
\end{figure}

While the techniques to find crystalline ordering have been non-specialised, we do not know what the crystal looks like. Knowing the structure of the crystal beforehand allows an enhancement of our tools to look specifically for the expected structure.

\section{What Crystals would Form}

In the interests of developing more specialised tools for detecting crystalline order, we need to understand the crystal structures that are likely to form. This involves finding the structures that we believe to be the lowest in energy. To find these lowest energy structures we need tools other than molecular dynamics since the timescales for molecular dynamics are beyond the limits of our simulations. To find a good approximation of the lowest energy structures we look to packing hard shapes, by modelling our molecules as hard shapes we are able to use an isopointal algorithm to get a good approximation of the closest packed configurations. These closest packed configurations were then used as the starting configuration for a Lennard-Jones system.

\begin{table}
    \caption{Tabulated energies}
\end{table}
The \scon molecule is an interesting case, the optimal packing of this shape is known as a compact packing. A compact packing is one where each pair of contacting particles has two other contacting particles in common, a phenomenon commonly found while packing discs. One of the consequences of a compact packing is that the packing of our \scon molecules is not unique.  Closest packed molecular configurations can be found by randomly arranging the bonds between the large and small particles in the compact packing. From testing the random arrangement has the same energy\tocheck as the crystalline structure however has far more entropy, giving it lower Gibbs free energy than the ordered structure. While from a theoretical perspective the random arrangement of bonds is more favourable, it is also more difficult to detect. Comparing the centers of mass of the ordered and disordered compact packings shows a dramatic difference in the degree of order we observe. The ordered packing has nicely ordered centers of mass that align with each other through the crystal, while the random arrangement shows centers of mass that appear highly disordered. In this case it is nessecary to move away from the description of molecules as a center of mass and an orientation and instead describe molecules as the individual particles.

In the case of \scon the move from a molecular to particle description allows us to correctly identify crystal structure regardless of the bonding of the molecules. The approach that works best for \sone is to count the number and type of neighbours of each particle. In the compact packing each small particle has three large and 1 small neighbour, and each large particle has four small and 3 large neighbours. This counting of neighbours correctly classifies the randomly oriented configuration as crystalline, while not returning false positives on the amorphous phase. Although counting the type and number of neighbours works well for the \sone molecule it is very specific in that it only works for this particular arrangement of particles. 


\section{Order Parameters}


Rather than dealing with each molecule as at the start of this chapter we are now exploring order parameters of each individual particle, rather than each molecule.

One simple order parameter is to take the radial distribution function and apply it to each particle. Using each particle is going to be much closer to something seen in a x-ray defractometer as it is focused on center of density, rather than center of mass. These function still display the short range order...


We can also look at other properties of each particle, like the number of neighbours.

With the techniques that we possess we are unable to find any significant crystalline order in the system. There are small regions which show order, a result expected for a distribution of all possible arrangements, however these small regions are static, not displaying an important aspect of a crystal; growth.



