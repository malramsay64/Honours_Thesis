\chapter{Structure}

In the previous chapter we investigated the dynamics of three molecular systems; \sone, \scon and \tri. The dynamics properties of these systems remained smooth throughout the range we simulated them, indicating that there was no phase transition taking place. While there is no obvious phase transition, one possible cause of the increasing time scales for the dynamics properties is the growth of crystal regions within the simulations.

The goal of this chapter is to investigate the structural properties of the three systems studied in the previous chapter; the \sone, \scon and \tri to determine whether the dynamic properties observed are purely dynamic properties or can be attributed to structural properties of the system.

\section{Is there Crystal Formation}

Over the course of the simulations to determine the dynamic properties of the system it is possible that the system transitioned from the purely amorphous phase to a structure that was a mixture of crystalline and amorphous regions. If these regions are long lived, rather than just fluctuations, as the temperature is dropped these regions are only more likely to grow; the lower entropy crystal phase is more favourable at low temperatures. Using this reasoning we can use the final configuration as an indication of the formation of crystalline regions over the entire temperature range of the simulation. From this final configuration we can take an inherent structure, removing any of the vibrational noise giving a configuration that is representative of the mean arrangement of the molecules. These arrangements are shown in \textfigref{inherent structures frame} where the colour of each molecule is given by its orientation.
\towrite{discuss inherent structures}

\begin{figure}
    \todofigure{Inherent structures}
    \caption{Inherent structures}
    \label{fig:inherent structures frame}
\end{figure}

While there appears to be no long range ordering in these configurations, to make a definitive statement we need more specialised tools. One of the alternate forms that could form is a plastic-crystal, a form where the position of the molecules is ordered while their orientations are disordered. This form can be more easily discerned by removing the orientational representation of the molecules, instead representing the center of mass of each molecule as a dot~\figref{inherent structures com}. This focuses the search for order, the centers of mass lining up over long ranges would be indicative of this plasti-crystal order.

\begin{figure}
    \todofigure{Inherent structure COM}
    \caption{COM of inherent structures}
    \label{fig:inherent structures com}
\end{figure}

While the techniques to find crystalline ordering have been non-specialised, we do not know what the crystal looks like. Knowing the structure of the crystal beforehand allows an enhancement of our tools to look specifically for the expected structure.

\section{What Crystals would Form}

In the interests of developing more specialised tools for detecting crystalline order, we need to understand the crystal structures that are likely to form. This involves finding the structures that we believe to be the lowest in energy, 






\section{Order Parameters}



