
\chapter{Introduction}

\section{Liquids}
\subsection{Structure}
Repulsive interactions give rise to short range order. Absence of long range order.
\subsection{Dynamics}


\section{Molecular Crystals}
\subsection{Structure}

One of the descriptions of the efficacy close packed crystal structure is the Molecular Coordination Number (MCN). This is very similar to the concept of coordination number in inorganic crystals however the generally much weaker interparticle interactions in molecules over ions introduce some nuances.

The typical picture of a molecular crystal is of a close packed structure~\cite{kitaigorodskii:73} where the crystal is arranged such that it occupies as much space as possible. This approach assumes the atoms behave in a manner similar to hard spheres from which molecular shapes can be constructed. The densest packing can the be found utilising one of a number of methods that search the configuration space. This method is suitably effective in inorganic chemistry~\cite{wells:73} however there are many organic crystals with a MCN = 14 which is not predicted using this model.

An alternate method for describing molecular crystals is based on the model of thinnest space covering~\cite{blatov:95} where the entire space is filled with Voroni-Dirichlet polyhedra. These polyhedra are formed by overlapping the atomic spheres until in such a way that they fill the entire space with minimum overlap. The vertexes of the polyhedra correspond to the intersections of the spheres and give information about the interatomic interactions of the molecule. The MCN of these convex polyhedra is the number of faces and the size of the face is proportional to the strength of the interaction. This description of molecular packing is based on the idea of soft deformable atoms and has been shown to be a better description of many molecular crystals~\cite{blatov:97,peresypkina:99,peresypkina:00}

The optimal packing of hard shapes has been a problem studied extensively throughout history, from circle packings studied by the ancient Greeks, to the packing of spheres studied by Kepler to more recently the packings of more complicated and unusual shapes~\cite{atkinson:12,torquato:12}. These problems of packing hard shapes however are incredibly difficult to solve analytically. The hexagonal close packing was first proposed as the most efficient packing of space in 1611~\cite{kepler:1611}, 400 years later Thomas Hales published the first proof~\cite{hales:05,hales:14} relying heavily on computers to check all possibilities. While a formal proof is difficult there are a number of computational methods that are commonly used in an attempt to find the global minimum of a function.


Long range order
\subsection{Formation}
\subsubsection{Thermodynamics}
\subsubsection{Crystal Growth}
\subsubsection{Nucleation}

\section{Supercooled Liquids}
Definition - cooled below melting point
Part of transition from liquid to solid - need to be supercooled for nucleation to take place.

Figure - temperature dependence of volume or enthalpy.
heat capacity of liquid greater than crystal - greater temperature dependence on enthalpy
Important temperatures

\subsection{Kinetics}
Viscosity - relation to relaxation time
Strong vs fragile liquids
Kauzmann temperature - entropy crisis
Rotational motion and viscosity proportional to viscosity - breaks down below 1.2 Tg
Heterogeneous dynamics - areas with motion, areas non-diffusing

\subsection{Fragility}

\subsection{Energy Landscape}
A crystal sampling an energy landscape
Amount of energy proportional to T
Higher T more of the space
When temperature lowered gets stuck in a local minimum
Slower cooling more likely to find lower energy minimum
Temperature affects how the system samples the landscape, not the landscape itself.

\subsection{Jamming}
Jammed when degrees of freedom removed
2 contacts remove DOF
Disc 2 DOF in 2D -> up/down, left/right
Snowman 3 DOF -> up/down left/right, rotation


\section{Glasses}
When the shear viscosity reaches a value (eg. \si{10^13} poise)
No phase transition
The structure of the glass is somewhat dependent on how it was constructed, the rate of cooling.
Although only by a small amount

\subsection{Structure}
Indistinguishable from that of a liquid.


\subsection{Formation}
Kinetically favoured product
Rate of change of volume or enthalpy with respect to temperature is on par with that of a solid

\section{Binary Mixtures}
Glass formers, separate to crystallise. Above 0.7? No increase in packing efficiency

\subsection{Metallic Glasses}
Of large interest, binary compound
Relatively easy to experiment with
Large array of phases - glassy and crystalline
Can have multiple crystal-glass phase transitions



\section{Previous Studies}
Wax cutouts paper - short range order

\section{Goals}

The goal of my project is to explore fundamental features of molecular shape; asymmetry and concavity, influence the properties of the various condensed phases; liquid, crystal, glass and supercooled-liquid. This requires the characterisation of a new set of molecular models for computer simulation. I will be addressing how the molecular orientation and transition motion couple during crystal growth, how the degree of concavity in the molecular shape determines the dynamics of rotations and translations in the low temperature liquid phase and, to find stable structures that determine the properties of crystals and glasses and how these structures are influenced by molecular shape.

