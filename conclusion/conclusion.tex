\chapter{Conclusion}

The molecules we have studied in this thesis show a diverse range of dynamic and structural behaviour. The small differences in shape among our three molecules resulted in liquid state dynamics that differed by over four orders of magnitude. We found that a large proportion of these differences in dynamics could be explained by the local structural rigidity. This local rigidity is a property directly related to the shape of the molecule, with highly concave molecules being more rigid than those with small concavities.

We also found that the shape of a molecule can strongly modify the crystal structures that form. In the case of the \dcon molecule an orientationally disordered crystal structure with formed the same underlying arrangement of particles as a binary compact packing of discs. This is a dramatic difference in structure compared to the strongly orientationally ordered crystal that the \done molecule freezes into.

We were able to explain the slow crystallisation of the \done molecule by the slow dynamics at the melting point. The additional entropy of the molecular rotations stabilises the liquid phase, lowering the melting point to a region in which the dynamics are incredibly slow. When comparing the rate of crystal growth to the structural relaxation time we see that crystallisation occurs with only very small molecular rearrangements and is therefore fast relative to the timescale of the molecular motion.

Finally we found a suitable candidate for simple single-component molecular glass former. The \tri molecule, even when presented with a liquid-crystal boundary, showed no sign of crystallising over many structural relaxation times. This will allow for the future investigations of simple molecular systems near the glass transition temperature.


\section{Future Work}

\begin{itemize}
    \item Further investigation of the local structural rigidity of molecules and how molecular shape influences the rigidity. Identify the properties of a shape that most strongly influence the local rigidity and how we can tune the rigidity.
    \item Investigate how an orientationally disordered crystal like that formed by the \dcon molecule influences properties of the solid phase.
    \item Use the \tri molecule as an example of a molecular glass former to investigate properties near the glass transition temperature.
\end{itemize}
