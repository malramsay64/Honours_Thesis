\chapter{Conclusion}

The molecules we have studies in this thesis show a diverse range of dynamic and structural behaviour. The small changes we made to the shapes of our molecules made sufficient changes to their local structural rigidity that the dynamic properties of the liquid changed by orders of magnitude.

Along with changes to the local rigidity in the amorphous phase, the local structure of the ordered phase can be dramatically changed with small modifications to the shape. Small changes to the shape of a molecule can dramatically affect the ordering in the crystal phase, with certain structures, for example \dcon, preferentially forming an orientationally disordered crystal structure.

The slow crystalline growth of molecular systems can be attributed to the slow dynamics at the melting point. The additional entropy of the rotations stabilise the liquid phase such that the dynamics at the melting point are slow. This means that while the rate of crystallisation is slow, relative to the structural relaxation time crystal growth is fast.

\section{Future Work}

\begin{itemize}
    \item A more detailed investigation of the coupling between the rotational and translational motion and how they influence overall dynamics.
    \item Further investigation of the local structural rigidity of molecules and how molecular shape influences the rigidity. Identify the properties of a shape that most strongly influence the local rigidity and how we can tune the rigidity.
    \item Investigate how an orientationally disordered crystal like that formed by the \dcon molecule influences properties of a solid.
\end{itemize}
