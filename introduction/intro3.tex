\documentclass{article}
\title{Untitled}
\author{Malcolm Ramsay}
\date{2015-2-23}
\begin{document}
\maketitle

At meny levels of chemistry you are told that there are three stte s of matter, solid, liquid and gas. In Physics you mightlearn about the more exotic states of plasma and bose einstein condensate, but you don't have to look past your smartphone so see states outside of those three you learn about. The screen of the phone is made of a glass, quite probably gorilla glass. This is used because of its hardness and imperviousness to scratched, but also because it is flexible, deviations in the structure are not going to lead to cracks, although with enough force it is still going to break. The screen itself is a is a LCD TN panel according to marketing speak, it translates to a liquid crytal display, twisted nematic panel. The nematic phase is found where there are rod like molecules which are roughly ordered in sheets but there is not distinct orientational order\tocheck. Then moving to the camera lens and home button which are covered in sapphire glass, a material even harder than gorilla glass but with many of the same properties. Corning (makers of gorilla glass, and corningware, and lab glass) recently ran an advertising campaign on youtube stating that we are in the glass age. In many ways they are true, the devices that we carry around with us are becoming smaller and smaller and demanyd ever stronger and more scratch resistant materials, even the materials that we have are not strong enough, how often do you see someone with a cracked sceen on their phone. 

Part of the reason that we don't currently have indestructible screens is a lack of understanding of the formation of glassy and crystal phases. We know that there are metallic and molecualr crystls that form, however we have no idea why there is are si\
My project aims to give insight into the formation of glasses and crystals, with particular whmphasis on molecules and the role of shape and concavites in the processes that drive these events. The process of forming a crystal or a glass requires paspanes - a mirror plane combined with a translation and screw axes - which is a rotation ombined with a translation. these symmetery operation have to be combined with all the possible uit cell shapes known as bravais lattices. 

In anuydiscussion of craystals we have to talk about  unit cells, these are the building lbocks of the crystal and we can describve a crystal by it's unit cell, eg \c{NaCl} type crystal or a \ce{zn2s} type crystal.\ce{CsCl}. When descibingh the unit cell one of the parameters thT WE USE IS THE SHAPE, a cubic, orthorhombic or hexagonal unit cell. gthese shapes are decsibed as \tocheck. The second desciptor is how the atoms are arraged with in the unit cell, know as bravis latticels . THe final desciptor of the unit cell is the symmetry operations that make  up, these are rotations ,importper rotations, screw axes, miglide planes that take the position of one atom and from the series of transformations create the position s of all the other atoms in the cell., Together these geometries, bravais lattices and synmetry operations describe the space group of a unit cell. The spacce grtoup is a systematic way of describing the a unit cell. It allows comparison of similar unit cells and makes up the smalles tdesciption of the unit cell. The space gourp and the position of one set of atoms withing the unit cell uis enough to completely recreate the unit cell. There are 219 of these spcae groups, 230 if you include chiral copies. 

\end{document}

