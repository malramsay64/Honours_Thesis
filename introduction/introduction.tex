\chapter {Introduction}

The crystallisation rates of molecular crystals are order of magnitudes slower than those of inorganic compounds with no full understanding of why this is the case. To give some insight into these processes we are reducing the problem to its simplest form, two dimensional molecules interacting on a plane. The simplicity of two dimensions allows a more intuitive visual study of the interactions and processes present in the system.

\section{Molecular Crystals}
\label{sec:molecular crystals}

To understand the formation of the crystal phase we first need an understanding of the crystal phase. The identifying feature of a crystalline system is the unit cell, the smallest periodic feature of a crystal. We describe the symmetry elements of a unit cell as a \emph{wallpaper group} allowing more direct comparisons of the crystal forms of each molecule.

\subsection{Identification of Crystals}

We need to be able to identify the crystal phase of the molecules. One method of measuring order is the \emph{radial distribution function} which provides information about the intermolecular distance of particles. The radial distribution function is found as
\begin{equation}
    G(r)
\end{equation}
 The reason we use the radial distribution function is used is that it can be linked the to the \emph{structure factor}, the quantity measured by x-ray diffraction, through a fourier transform
 \begin{equation}
     S(q)
 \end{equation}
The radial distribution shows the extent of ordering in a system, a liquid phase will show strong short range ordering with a random distribution of molecules past the first or second shell of molecules. Crystalline systems will show sharp distinct peaks over long ranges, with systems in between exhibiting some degree of ordering.

\section{Liquid State Dynamics}

In understanding the rate of crystallisation of a liquid we also need to understand the dynamics of the system. Knowing the timescales on which molecules move give some indication of the timescales on which crystallisation will take place.

The dynamic quantities that we want to measure are the Diffusion constant, rotational relaxations time, and structural relaxation time. These quantities are defined in \textchapref{dynamics} and are chosen for their relation to measurable properties in real world experimental systems. The Diffusion constant $D$ is closely related to the viscosity $\eta$ by the Stokes-Einstein equation
\begin{equation}
    D =
\end{equation}
The rotational relaxation time is a quantity that can be measured using light scattering experiments, while the structural relaxation time can be measured using NMR. Measuring quantities that are comparable to real experimental system allow investigation of results found in those systems and for any understanding that comes from the simulated system to be tested by experiment.

\section{The Complications of Supercooled Liquids}

The supercooled liquid phase is structurally identical to the liquid phase, differing only in being below the melting point of the crystal. The supercooled liquid phase is important in the formation of the solid phase, however the properties of the supercooled liquid phase are potentially pose problems in simulations.

\subsection{Nucleation}

At the melting point both the liquid and the crystal phases have the same free energy, in the NPT thermodynamic system this is $\Delta G$. Despite the two phases having the same energy the nucleation event required for a crystal to grow is incredibly unlikely at the melting point. The interface between the liquid and the crystal has a higher energy than the either the liquid or crystal phases and so when some crystal nucleates it will quickly melt again. For the initial nucleus to become stable we need a crystal phase that is more stable than the liquid to make up for the extra energy at the liquid-crystal boundary. This means moving to the supercooled liquid phase. The greater the supercooling of a liquid the more favourable nucleation is, however there is competition with the dynamics, lower temperatures mean slower dynamics.

\subsection{Fragility}

Fragility is the description of the behaviour of supercooled liquids near the glass transition temperature. The dynamic quantities of liquids exhibit an Arrhenius temperature dependence, a relation of the form
\begin{equation}
    D
\end{equation}
where ... In supercooled liquids this type of behaviour is displayed by strong liquids like silica. However many molecular system exhibit non-arrhenius behaviour where the dynamics scale faster than arrhenius and are known as fragile. This non-arrhenius behaviour presents a problem for simulations as the timescales that we are able to investigate in simulations is limited. This is likely to be a problem since many fragile liquids are molecular in nature.


\subsection{Glass Formation}

Glasses are an amorphous solid phase formed by cooling a liquid quickly such that it does not have a chance to nucleate and form the crystal phase. The rate of cooling that is considered quickly is different for each liquid and is related to the dynamic properties of the liquid. Slow cooling requires the liquid to relax at each temperature, this can become a problem in fragile liquids where the relaxation time is increasing faster than exponentially. Cooling the system to a temperature at which glass formation could take place might be longer than we can run simulations.
