
\chapter{Introduction}

\section{Liquids}
The liquid phase is the most prevalent on Earth, from the liquid water in our oceans, to the liquid magma beneath the crust, it plays an important role in keeping us alive, but was also just as important in our evolutionary history.

The liquid phase can be characterised by its near incompressibility, and its ability to flow. \towrite{General description of liquid phase}

The liquid phase exists due to an attractive force that extends far enough~\figref{fig:potentials} for a particles to move past one another while still feeling the potential~\cite{tejero:94}. When the potential well becomes too narrow the liquid phase ceases to exist leaving only the solid and gas phases.

\begin{figure}
    \todofigure{Potentials, short and long range}
    \label{fig:potentials}
    \caption{Short range potential ($\sigma < 1.60$) compared to a long range potential ($\sigma > 1.60$)}
\end{figure}

\subsection{Structure}
The structure of the liquid is defined by this attractive force, with particles forming distinct shells~\figref{fig:radial distribution}. These shells however decay over distance, leaving an isotropic distribution. The \emph{radial distribution function} $g(r)$ is the distribution of particles found at a distance $r$ from any other particle,
\begin{align}
    g(r) &= \frac{1}{\rho} \langle \delta(r - r_i) \rangle
\end{align}
where $\rho = \frac{N}{V}$, the number density. The radial distribution function is a factor that when multiplied by the bulk density $\rho$ we get a local density $\rho(r) = \rho g(r)$. This means we can integrate over all space,
\begin{equation}
    \int_0^\infty \rho g(r) 4 \pi r^2\,dr = N-1 \approx N
\end{equation}
and get the total number of particles in the system.

The radial distribution function is also crucial in comparing simulation to experiment, x-ray diffraction patterns are also the result of two particle interactions and can be used to determine the short range order of the liquid, giving peaks just like the radial distribution function.

Form well defined shells of nearest neighbours~\cite[p2]{barrat:03}
\begin{figure}
    \todofigure{Radial distribution function for liquid}
    \label{fig:radial distribution}
    \caption{We can see the short range order which after a few shells decays to an isotropic distribution.}
\end{figure}

\subsection{Thermodynamics}
Stabilised by extra degrees of freedom of molecule.

\subsection{Dynamics}
The motion of particles in a liquid can best be explained by Brownian motion, where the path taken by an atom appears to be random and irregular. We can describe this mathematically using the \emph{Langevin equation};
\begin{equation}
    \ddiff{\vect u}{t} = - \zeta \vect u + \vect A(t)
\end{equation}
There are two parts to this equation, the frictional force ($\zeta$) generating the drag of the particle in the medium. There is also the stochastic fluctuating force ($A(t)$) which arises from the frequent collisions of particles in the liquid phase.

This can be solved~\cite{mcquarrie:75} by looking at the ensemble averages of the motion, giving two results. When the time $t$ is very small we get;
\begin{equation}
    \langle | \vect r - \vect{r_0}|^2 \rangle \rightarrow |u_0^2|t^2,   t \rightarrow 0
\end{equation}
while on large time scales we have a different result;
\begin{equation}
    \langle | \vect r - \vect{r_0} | \rangle = \frac{6kT}{m\zeta} t = 6Dt
\end{equation}
where we have introduced the diffusion constant $D = kT/m\zeta$. This tells us that the mean squared displacement ($\langle | \vect r - \vect{r_0} |^2 \rangle$) at short time scales will have a $t^2$ relationship, however after a long time compared to $\zeta^{-1}$ the statistical nature of the motion takes over and we see linear dependence on $t$~\figref{fig:MSD}.

\begin{figure}
    \todofigure{MSD plot showing both short and long timescale dependence}
    \label{fig:MSD}
    \caption{Calculation illustrating the time dependence of the mean squared displacement}
\end{figure}

\towrite{Rotational relaxations}

\section{Supercooled Liquids}

Intrinsically there is nothing to differentiate a liquid from a supercooled liquid, the difference comes from their relationship to the melting point. Liquids are the liquid phase when the temperature is above the melting point, supercooled liquids have a temperature below the melting point. The process of supercooling a liquid is an integral part of forming the solid phase, spontaneous crystallisation takes a long time to occur at the melting point due to the large surface area of any nucleus~\figref{fig:supercool crys}.

\begin{figure}
    \todofigure{Rate of crystallisation as a function of temperature}
    \caption{There is a supercooling that gives maximum rate of crystal formation}
    \label{fig:supercool crys}
\end{figure}

\begin{figure}
    \todofigure{Temperature dependence of enthalpy or volume}
\end{figure}
heat capacity of liquid greater than crystal - greater temperature dependence on enthalpy~\cite{debenedetti:01}
Important temperatures

\subsection{Kinetics}
Viscosity - relation to relaxation time
Strong vs fragile liquids
Kauzmann temperature - entropy crisis
Entropy of liquid decreases faster with T than in a solid.
Rotational motion and viscosity proportional to viscosity - breaks down below 1.2 Tg
Heterogeneous dynamics - areas with motion, areas non-diffusing

\subsection{Fragility}
Super-Arrhenius behaviour.
Rotational motion decays faster than translational motion

\subsection{Energy Landscape}
A crystal sampling an energy landscape
Amount of energy proportional to T
Higher T more of the space
When temperature lowered gets stuck in a local minimum
Slower cooling more likely to find lower energy minimum
Temperature affects how the system samples the landscape, not the landscape itself.

\subsection{Jamming}
Jammed when degrees of freedom removed
2 contacts remove DOF
Disc 2 DOF in 2D -> up/down, left/right
Snowman 3 DOF -> up/down left/right, rotation

\subsection{Thermodynamics}


\section{Molecular Crystals}
\subsection{Structure}

One of the descriptions of the efficacy close packed crystal structure is the Molecular Coordination Number (MCN). This is very similar to the concept of coordination number in inorganic crystals however the generally much weaker interparticle interactions in molecules over ions introduce some nuances.

The typical picture of a molecular crystal is of a close packed structure~\cite{kitaigorodskii:73} where the crystal is arranged such that it occupies as much space as possible. This approach assumes the atoms behave in a manner similar to hard spheres from which molecular shapes can be constructed. The densest packing can the be found utilising one of a number of methods that search the configuration space. This method is suitably effective in inorganic chemistry~\cite{wells:84} however there are many organic crystals with a MCN = 14 which is not predicted using this model.

An alternate method for describing molecular crystals is based on the model of thinnest space covering~\cite{blatov:95} where the entire space is filled with Voroni-Dirichlet polyhedra. These polyhedra are formed by overlapping the atomic spheres until in such a way that they fill the entire space with minimum overlap. The vertexes of the polyhedra correspond to the intersections of the spheres and give information about the interatomic interactions of the molecule. The MCN of these convex polyhedra is the number of faces and the size of the face is proportional to the strength of the interaction. This description of molecular packing is based on the idea of soft deformable atoms and has been shown to be a better description of many molecular crystals~\cite{blatov:97,peresypkina:99,peresypkina:00}

The optimal packing of hard shapes has been a problem studied extensively throughout history, from circle packings studied by the ancient Greeks, to the packing of spheres studied by Kepler to more recently the packings of more complicated and unusual shapes~\cite{atkinson:12,torquato:12}. These problems of packing hard shapes however are incredibly difficult to solve analytically. The hexagonal close packing was first proposed as the most efficient packing of space in 1611~\cite{kepler:1611}, 400 years later Thomas Hales published the first proof~\cite{hales:05,hales:14} relying heavily on computers to check all possibilities. While a formal proof is difficult there are a number of computational methods that are commonly used in an attempt to find the global minimum of a function.

Molecular crystals typically have simple unit cells~\cite{brock:94}. Periodic lattice

Long range order
\subsection{Formation}
\subsubsection{Thermodynamics}
First order phase transition

\subsubsection{Crystal Growth Kinetics}
Rate of crystal growth determined by the rate of nucleation and by the speed at which the crystal-liquid interface advances~\cite{turnbull:69}.
Recalescence - latent heat of crystallisation, results in single nucleation site.~\cite{turnbull:69} Most effective seed first, afterwards temperature increases preventing any further nucleation (any other seed is worse, since it hadn't already formed a crystal)

Anisotropic - some faces crystallise more easily, this gives rise to rod shaped crystals.
The larger the face the slower the crystal growth along that face.

Control of growth~\cite{de-yoreo:03}

A series of steps containing kinks~\cite{chernov:61} More bonds to neighbours when attaching, more likely to stick at these locations. Growth rate impacted by the number of these kink sites.

There is a point at which the driving force for crystallisation, which increases with decreasing temperature and the atomic mobility, which decreases with increasing temperature results in a minimum in the time for a volume fraction to crystallise.~\cite{uhlmann:72}
\begin{figure}
    \todofigure{Time temperature transformation curve}
\end{figure}

The crystal growth a large supercoolings is dependent on the fragility of the liquid~\cite{ediger:08}. This is in the region where the crystal growth is limited by diffusion. 


\subsubsection{Nucleation}
Requires passage over a free energy barrier. The free energy per molecule of a crystal phase is only lower than the liquid in the bulk phase; the surface is less well bound. There is such a size where the free energy of a system will decrease whether the nucleus grows or dissolves, a transition state/saddle point.

Nucleation can be controlled by manipulating the surface energy of the nucleus.~\cite{de-yoreo:03}
 
Rate limiting step, needs to be supercooled
 
\section{Glasses}
When the shear viscosity reaches a value (eg. \si{10^13} poise)
No thermodynamic phase transition~\cite{santen:00}
The structure of the glass is somewhat dependent on how it was constructed, the rate of cooling, although only by a small amount
Glasses form from in every category of material~\cite{turnbull:69}

Have a number of useful properties~\cite{greer:07}

Indistinguishable from that of a liquid on thermodynamic grounds~\cite{santen:00}

There are some glasses that have formed from every bond type: covalent, van der Waals, ionic, metallic, hydrogen.~\cite{turnbull:69}

\subsection{Structure}


\subsection{Formation}
Kinetically favoured product
Rate of change of volume or enthalpy with respect to temperature is on par with that of a solid
Ideal glass - lowest energy non-crystalline phase

The most favourable conditions for glass formation involve a large viscosity at the melting point and a rapidly rising viscosity below the melting point.~\cite{uhlmann:72}



The value of the maximum crystal growth rate might be correlated with glass forming ability~\cite{tang:13}

Little difference between activation energies in poor vs strong glass formers~\cite{tang:13}

Locally ordered clusters key to formation~\cite{yang:12} Higher local packing efficiency - lower energy. 

\section{Energy Landscape}
One way of understanding the behaviours system is using a multidimensional topographic map, with a dimension for each degree of freedom of each particle. The minima of this topology correspond to regions where there is little net force on each particle, they are also known as inherent structures. With enough energy the system can sample all configurations, this corresponds to the liquid phase. Reducing the energy we begin to limit the configurations the system can sample. If this occurs slowly enough the system will find the lowest energy state, the crystal. However if the energy drops below that of the transition state to the crystal, then the best the system can do is to form a glass.~\cite{stillinger:95} Ideal glass is lowest energy non-crystalline state.

\begin{figure}
    \todofigure{2D energy landscape}
\end{figure}

Fragile glass formers have two types of relaxations, dual scale topology. $\alpha$ relaxations large wells, $\beta$ relaxations small wells.~\cite{stillinger:95}

\section{Binary Mixtures}
Glass formers, separate to crystallise. Above 0.7? No increase in packing efficiency

\subsection{Metallic Glasses}
Of large interest, binary compound
Relatively easy to experiment with
Large array of phases - glassy and crystalline
Can have multiple crystal-glass phase transitions



\section{Previous Studies}
Wax cutouts paper - short range order

\section{Goals}

The goal of my project is to explore fundamental features of molecular shape; asymmetry and concavity, influence the properties of the various condensed phases; liquid, crystal, glass and supercooled-liquid. This requires the characterisation of a new set of molecular models for computer simulation. I will be addressing how the molecular orientation and transition motion couple during crystal growth, how the degree of concavity in the molecular shape determines the dynamics of rotations and translations in the low temperature liquid phase and, to find stable structures that determine the properties of crystals and glasses and how these structures are influenced by molecular shape.

