
\chapter{Introduction}


Manipulating the formation of a condensed phase is a critical part of both nature and technology. From molluscs controlling crystal formation\tocite, to forming molecular glasses to increase the solubility of drugs\tocite, to the plastics we use everyday\tocite, silicon in photovoltaic cells\tocite and many other materials that we rely on every day. As the devices we use become smaller the need to understand the processes that form them becomes more important. A complete theory of glass formation is still elusive and our ability to control crystal formation is far from what is available to nature. To better understand the formation of the solid phase an understanding of the molecular rearrangements that take place as we move from a molecular liquid through a supercooled liquid to either a glass or molecular crystal.

\section{Molecular Crystals}

It might be expected that the diversity in molecular shape will correspond to a diversity in molecular crystal structure. This is not the case, \SI{75}{\percent} of molecules occupy only five space groups\tocite. These space groups are a series of rotations, reflections improper rotations, screw axes and glide planes that completely describe the unit cell of a crystal. There are 219 space groups in three dimensional space (230 including chiral copies) making the distribution of space groups within molecular crystals significant.

While the ability to identify the space group of a molecule is useful, more useful is the ability to predict the space group a molecule will occupy. One method is to assume the molecules are hard shapes and to find the arrangement with the largest volume of space occupied\tocite, also known as the \emph{packing fraction}.  The simplest example of this is packing spheres to which Kepler proposed the hexagonal closed packed structure~\figref{} in 1611\tocite. It has taken over 400 years to prove\tocite this arrangement gives the largest packing fraction, requiring computers to check every possible configuration in space. Packing circles onto a plane is a significantly easier problem to prove\tocite and to visualise\figref{}. The visualisation is important to understanding new molecular behaviour, following a process on a plane is much easier than in space. The reduction in dimensionality also simplifies the space groups, called \emph{wallpaper groups} in two dimensions. There are only 17 wallpaper groups\figref{} and of these only the pg and the p2gg wallpaper groups represent nearly \SI{95}{\percent} of 2D molecular crystals\tocite. For shapes to 

\towrite{concave particles}

The periodic and ordered structure that comes from the symmetry of the space group is also responsible for a number of properties of the crystal phase. \towrite{properties of the crystal phase}
For some applications the solid form is necessary however there are unfavourable properties of the crystal, an example of this is in drug design, the drug administered as a tablet needs to be solid, however it also needs to be soluble\tocite or fiber optics where the fiber needs to be solid but also needs to be flexible\tocite. These applications can benefit from an amorphous solid, also known as a glass.

\section{Molecular Glasses}

The structure of a glass is indistinguishable from that of a liquid~\figref{}. Despite the liquid like structure the glassy phase is most definitely solid~\tocite contrary to urban legends that it can flow over long timescales~\tocite. Much of the misunderstanding of the glassy phase is related to the lack of a first order phase transition~\tocite, none of the thermodynamic properties change upon transition. Instead the \emph{glass transition temperature} (\si{\Tg}) is defined as the temperature at which the viscosity reaches \SI{e13}{\poise}, a somewhat arbitrary value. However as we approach the \si{\Tg} from the liquid there is a dramatic increase in the viscosity~\figref{}, especially for \emph{fragile} liquids.

Glass formers are characterised by their behaviour near \si{\Tg}. Liquids that adhere to a purely Arrhenius regime are considered \emph{strong} glass formers, the typical example being silica \ce{SiO2}. Those that have super-Arrhenius behaviour are considered \emph{fragile}, with \emph{o}-terphenyl being the canonical example. Systems that display an Arrhenius temperature dependence,
\begin{equation}
    k = A \e^{-E_a/(RT)}
\end{equation}
have an energy barrier that requires an activation energy ($E_a$) to proceed. This activation energy remains constant throughout the temperature range. Fragile liquids however have an activation energy that depends on the temperature\tocite, characterised by large increases in viscosity over a small temperature range as \si{\Tg} is approached. The temperature dependence of the viscosity for fragile liquids indicates that the glass transition temperature is more than just an arbitrary value, rather; the glass transition is an inherent characteristic of a material.

Developing a theoretical understanding of a material requires a model system, something simple enough to model easily, yet easy enough to scale up. A defining feature of fragile glass formers~\tabref{} is the prominence of molecules, 


\begin{figure}
    \todofigure{Plot of entropy vs Temperature, where all the phases lie}
    \caption{}
    \label{fig:entropy}
\end{figure}

\section{Project Goals}

The goal of my project is to explore fundamental features of molecular shape; asymmetry and concavity, influence the properties of the various condensed phases; liquid, crystal, glass and supercooled-liquid. This requires the characterisation of a new set of molecular models for computer simulation. I will be addressing how the molecular orientation and translational motion couple during crystal growth, how the degree of concavity in the molecular shape determines the dynamics of rotations and translations in the low temperature liquid phase and, to find stable structures that determine the properties of crystals and glasses and how these structures are influenced by molecular shape.

