
\chapter{Introduction}


Manipulating the formation of a condensed phase is a critical part of both nature and technology. From molluscs controlling crystal formation\tocite, to forming molecular glasses to increase the solubility of drugs\tocite, to the plastics we use everyday\tocite, silicon in photovoltaic cells\tocite and many other materials that we rely on every day. As the devices we use become smaller the need to understand the processes that form them becomes more important. A complete theory of glass formation is still elusive and our ability to control crystal formation is far from what is available to nature. To better understand the formation of the solid phase an understanding of the molecular rearrangements that take place as we move from a molecular liquid through a supercooled liquid to either a glass or molecular crystal.

\section{Molecular Crystals}

It might be expected that the diversity in molecular shape will correspond to a diversity in molecular crystal structure. This is not the case, \SI{75}{\percent} of molecules occupy only five space groups\tocite. These space groups are a series of rotations, reflections improper rotations, screw axes and glide planes that completely describe the unit cell of a crystal. There are 219 space groups in three dimensional space (230 including chiral copies) making the distribution of space groups within molecular crystals significant.

The ability to identify the space group of a molecule is useful, more useful is the ability to predict the space group a molecule will occupy. One method is to assume the molecules are hard shapes and to find the arrangement with the largest volume of space occupied\tocite, also known as the \emph{packing fraction}.  The simplest example of this is packing spheres to which Kepler proposed the hexagonal closed packed structure~\figref{} in 1611\tocite. It has taken over 400 years to prove\tocite this arrangement gives the largest packing fraction, requiring computers to check every possible configuration in space. Packing circles onto a plane is a significantly easier problem to prove\tocite and to visualise\figref. The visualisation is important to understanding new molecular behaviour, following a process on a plane is much easier than in space. The reduction in dimensionality also simplifies the space groups, called \emph{wallpaper groups} in two dimensions. There are only 17 wallpaper groups\figref and again only a few groups give closest packed structures of molecules\tocite.

The periodic and ordered structure that comes from the symmetry of the space group is also responsible for a number of properties of the crystal phase. \towrite{properties of the crystal phase}
For some applications the solid form is necessary however there are unfavourable properties of the crystal, an example of this is in drug design, the drug administered as a tablet needs to be solid, however it also needs to be soluble\tocite or fiber optics where the fiber needs to be solid but also needs to be flexible\tocite. These applications can benefit from an amorphous solid, also known as a glass.

\section{Molecular Glasses}

The glassy state is present in materials with all bond types; covalent, van der Waals, ionic, metallic and hydrogen\tocite.


\begin{figure}
    \todofigure{Plot of entropy vs Temperature, where all the phases lie}
    \caption{}
    \label{fig:entropy}
\end{figure}

\section{Project Goals}

The goal of my project is to explore fundamental features of molecular shape; asymmetry and concavity, influence the properties of the various condensed phases; liquid, crystal, glass and supercooled-liquid. This requires the characterisation of a new set of molecular models for computer simulation. I will be addressing how the molecular orientation and translational motion couple during crystal growth, how the degree of concavity in the molecular shape determines the dynamics of rotations and translations in the low temperature liquid phase and, to find stable structures that determine the properties of crystals and glasses and how these structures are influenced by molecular shape.

