\chapter{Crystallisation}

In this chapter we apply the concepts of the previous two chapters together, in other words examining the dynamic structural changes present as a liquid moves from the liquid to the crystalline phase. Since the molecules \sone, \sone and \tri were chosen partly because of their lack of crystallisation the timescales on which these dynamics are studied are exceptionally long.


\section{Melting Points}

The most important temperature for crystal formation is the melting point, try to form a crystal above it and nothing is going to happen, too low and the dynamics are too slow. Getting an idea of the melting point can be done by heating the crystal until it disorders. The melting points and molecular energy for the \sone and \scon are tabulated in \texttabref{melting sone, melting scon} respectively. The energy per molecule was determined by equilibrating the crystal below the melting point and taking an inherent structure. The melting points were determined by visual inspection of the resulting inherent structures at each temperature, when there was a region of disorder the temperature was deemed to be above the melting point.

\begin{table}
    \begin{tabular}{r | c c }
        Space Group & Melting Point & Energy per molecule \\
        p2 & & \\
        p2gg & & \\
        p3 & & \\
        p1 & & \\
    \end{tabular}
    \caption{Melting points for \sone}
    \label{tab:melting sone}
\end{table}

\begin{table}
    \begin{tabular}{r | c c }
        Space Group & Melting Point & Energy per molecule \\
        p2 & & \\
        p2gg & & \\
        p3 & & \\
        p1 & & \\
    \end{tabular}
    \caption{Melting points for \scon}
    \label{tab:melting scon}
\end{table}

Determining melting points in this way tends to overestimate the melting point, areas of melted crystal are highly unfavourable and there is no surface on which the melting can occur.

\section{Crystallisation at a Boundary}

Like melting from a bulk crystal, the formation of a crystal in a bulk liquid is difficult, the nucleation event is the process that limits the formation. The solution to both these issues is to create an initial state that is half liquid and half crystal. At temperatures above the melting point the liquid region will advance into the crystal region, below the melting point the crystal should advance on the liquid region. The rate of procession of the crystal into the liquid can be used to determine a crystal growth rate for that temperature.

\subsection{\sone}

The closest packed wallpaper group for the \sone molecule is a close race between the p2, p2gg and p3 groups\tocheck. The structures of all these crystals are very different, making use of different interactions~\figref{sone closest packed}. We hypothesised that the concavities play a role in the stability of the liquid phase, the crystal with these dimeric interactions is the p2gg crystal and so will be most likely to form.

\begin{figure}
    \begin{subfigure}{0.3\textwidth}
        \includegraphics[width=\linewidth]{{{Snowman-0.637556-1.00-p3}}}
        \caption{p3}
        \label{fig:sone p3}
    \end{subfigure}
    \begin{subfigure}{0.3\textwidth}
        \includegraphics[width=\linewidth]{{{Snowman-0.637556-1.00-p2gg}}}
        \caption{p2gg}
        \label{fig:sone p2gg}
    \end{subfigure}
    \begin{subfigure}{0.3\textwidth}
        \includegraphics[width=\linewidth]{{{Snowman-0.637556-1.00-pg}}}
        \caption{pg}
        \label{fig:sone pg}
    \end{subfigure}
    \caption{Closest packed crystal structures of \sone}
    \label{fig:sone closest packed}
\end{figure}

This brought the approximation of the melting point down to $0.68$, however more interesting than the new melting point is a solid state phase transition that is occurring in the crystal region~\figref{sone solid phase}. Along with a melting of the crystal phase there are regions that have reoriented. This tells us that our guess of the appropriate space group was incorrect. However somewhat serendipitously we managed to achieve nucleation while running long simulations for the dynamics, growing into a number of orientationally distinct regions~\figref{sone crystal}. This new crystal that has formed was not even considered previously, it is 5th\tocheck in term of packing fraction. 

\begin{figure}
    \centering
    \includegraphics[width=\textwidth]{{{Snowman-0.70-0.637556-1.0-p2gg-1-frame}}}
    \caption{Two phase simulation of \sone}
    \label{fig:sone solid phase}
\end{figure}

\begin{figure}
    \centering
    \includegraphics[width=\textwidth]{{{Snowman-0.65-0.637556-1.0-frame}}}
    \caption{Crystallisation}
    \label{fig:sone crystal}
\end{figure}

So why is the p2mg crystal more favourable than other crystals with a higher packing fraction. One possibility is that the packing of hard discs is not a suitable approximation for this system. There is some research~\cite{medvedev:87,blatov:95,blatov:97} that suggests that hard discs are not suitable for soft three dimensional systems because they give a different number of neighbours than experimental results. For our systems hard and soft discs both give the same number of neighbours. The other possible explanation is that at high temperatures it becomes unfavourable for molecules to interlock in their concavities as this prevents the molecular vibrations. The p2mg structure that grows has room for vibrational and rotational motion, there is also enough flexibility in the structure for packing defects, the most common being a molecule oriented in the wrong direction.

\subsection{Crystal Growth Kinetics}

Now we have a stable crystal structure that forms in a reasonable time we can perform crystal growth kinetics. 

\towrite{Crystal growth kinetics data, regio plots, regions become far less dynamic, crystal growth rate as a function of temperature}

\section{\scon}


