\chapter{Crystallisation}

In this chapter we apply the concepts of the previous two chapters together, in other words examining the dynamic structural changes present at the phase transition from the liquid to the crystal phase. Since one of the criteria for the choice of molecule is that they are resistant to crystallisation the timescales that these dynamics occur on are exceptionally long.

\section{Melting Points}

After finding the lowest energy crystal structures and the dynamical properties of the molecules the remaining property for the study of crystallisation dynamics is the melting point. We need to know the melting point to get a range of temperatures at which crystal formation is expected to take place. Melting points were found by heating up from the crystal phase until it starts disordering. This gives an upper estimate on the melting point, similarly to the nucleation events required for crystallisation a nucleation event is required for melting to occur. The melting points obtained by this method are:
\begin{itemize}
    \item \sone 0.80
    \item \scon 1.90
    \item \tri 1.40
\end{itemize}

\section{Crystallisation Dynamics}

\subsection{Two Phase Systems}

To get around the issues of waiting for a nucleation event to occur to see the growth of a crystal, we can create a two phase system in which the liquid-crystal boundary is already present. This can then be used to observe the growth of the crystal phase on this surface. 

In the \sone system we see very little growth.

Same for the \scon system

\tri system see no growth

\subsection{Spontaneous Nucleation}



\section{Time Dependence of Structural Properties}


