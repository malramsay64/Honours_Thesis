\chapter{The Dynamics of Molecular Liquids}
\label{sec:dynamics}

The motion of a molecule is generally comprised of two components, the translational motion comprised of movements of the center of mass, and rotational motion comprised changes in the orientation of the molecule. In a liquid, both translational and rotational motions will involve some degree of cooperative motion with the neighbouring molecules. The nature of the coupling is different for each type of motion. Translational motion requires a vacant space to move into often giving rise to string like coupling~\tocite. Rotational motion is more dependent on the shape of the molecule, molecules close to circular are going to be able to rotate more easily than an elongated rod. In this Chapter we want to investigate the influence of molecular shape on the coupling of the rotational and translational motions as well as the role of molecular shape in the overall dynamics. To begin we must introduce some model molecules a choice we describe in \textsecref{mol choice}. Next, we shall introduce the various dynamic quantities we will be using to characterise the motion of the molecules in the liquid state \secref{dynamic quantities}. In the subsequent Sections we present our analysis of the dynamics of our model molecules.


\section{Choice of Molecules}
\label{sec:mol choice}

Our goal in studying these molecular systems is not to reproduce the behaviour of a specific molecule, but rather to explore how basic features of all molecules, namely shape, influence the molecular motion in the liquid phase. Our choice of molecules is guided by the following requirements:
\begin{itemize}
    \item Each of our model molecules should capture recognisable features of real molecules, in particular anisotropy and concavity.
    \item Each model molecule should show evidence of slow crystallisation kinetics. Crystallisation in 2D is very rapid for circular discs, to characterise dynamics below the melting point we need to inhibit crystallisation.
    \item The model molecule should be simple to compute since we will have to carry out long simulated trajectories.
\end{itemize}

With little prior research into the properties of molecular systems we chose the simplest types of molecules possible, a two particle dimer~\figref{dimer} and a three particle trimer~\figref{trimer}. The simplicity of these molecules allow us to survey a range of different molecular parameters to find molecules of interest for further study in this thesis. The survey included both the Dimer and Trimer molecule types over a radius range $r = [0.5,1]$, a distance range of $d = [1,1+r]$ and an angle range of $\theta = [90^\circ,180^\circ]$. These ranges were chosen so the resulting molecules are sufficiently distinct from a disc, a problem that has been studied for a long time~\cite{verlet:67}.

\begin{figure}
    \centering
    \begin{subfigure}[t]{0.48\textwidth}
        \includegraphics[width=\linewidth]{dimer}
        \caption{The Dimer molecule consists of a large particle of radius $1.0$, a small particle of radius $r$ separated by a distance $d$.}
        \label{fig:dimer}
    \end{subfigure}\hfill
    \begin{subfigure}[t]{0.48\textwidth}
        \includegraphics[width=\linewidth]{trimer}
        \caption{The Trimer molecule is comprised of a large particle of radius $1.0$ which subtends an angle $\theta$ between two small particles of radius $r$ at a distance $d$.}
        \label{fig:trimer}
    \end{subfigure}
    \caption{Construction of the molecules used in this thesis.}
    \label{fig:construction}
\end{figure}

The resulting collection of 50 molecules was cooled from a randomly generated high temperature state, resulting in a low temperature structure independent of the initial state. We assessed these low temperature structures for suitability on the criteria listed above, with one addition; the molecules had to be similar enough such that we could attribute any differences in dynamics to the shape of the molecule.

Discs with a radius $r=0.637556$ and $r=1.0$ have an interesting property, they can form an efficient binary packing of space, known as a compact packing~\appref{}. A dimer with a radius $r= 0.637556$ and a distance between the molecular centers $d= 1.637556$~\figref{dcon} hereafter denoted \dcon has a crystal structure that mirrors this compact packing. Despite \dcon having this well packed crystal structure, the low temperature state exhibited none of this crystal, indicating it is resilient to crystal formation. Retaining a radius of 0.637556 for the other molecules allows a more direct comparison, any differences are due to properties of the arrangement of particles, their shape, rather than the size ratios of the constituent particles. A distance $d= 1.0$ was chosen for another dimer~\figref{done} denoted \done and also for a trimer~\figref{tri} denoted \tri, for both consistency between the molecules and to allow all molecules to be described radially. An angle $\theta=120^\circ$ was chosen as this angle allows the maximum number of interparticle interactions for a trimer.

\begin{figure}
    \centering
    \captionsetup{justification=centering}
    \begin{subfigure}[t]{0.3\textwidth}
        \includegraphics[width=\linewidth]{{{Sone}}}
        \caption{$r=0.637556,$\\$d=1.0$ \\\done}
        \label{fig:done}
    \end{subfigure}\hfill
    \begin{subfigure}[t]{0.3\textwidth}
        \includegraphics[width=\linewidth]{{{Scon}}}
        \caption{$r=0.637556,$\\$d=1.637556$ \\\dcon}
        \label{fig:dcon}
    \end{subfigure}\hfill
    \begin{subfigure}[t]{0.3\textwidth}
        \includegraphics[width=\linewidth]{{{Tri}}}
        \caption{$r=0.637556,$\\$d=1.0, \theta=120 $\\\tri}
        \label{fig:tri}
    \end{subfigure}
    \caption[Molecules chosen for further study]{The molecules chosen for study in this thesis in order of the size of the concavity. The \done with the smallest concavity, \dcon with the smaller disc contacting the large disc and \tri which is a very similar to the \done molecule.}
    \label{fig:my mols}
\end{figure}

\section{Quantifying Dynamics in a Liquid}

There are three aspects of dynamics that we will measure in our liquids, translational motion, rotational relaxation and structural relaxation. Each of these aspects is directly related to an experimentally observable property, allowing the comparison of these results with real world experiments.

For translational motion we shall determine the time dependence using the Mean Squared Displacement (MSD) of the centers of mass~\figref{msd ex}, given by
\begin{align}
    MSD(t) &= \langle \delta (t)^2 \rangle,\\
    \delta(t) &= \sqrt{(x(t) - x_0)^2 + (y(t) - y_0)^2}
    \label{eq:msd}
\end{align}
where $x(t)$ and $y(t)$ are the coordinates of the COM at time $t$, $x_0$ and $y_0$ are the initial coordinates of the molecule, and the angle brackets $\langle\,\rangle$ denote averaging over all molecules. The MSD is a quantity calculated at each point in time and so not particularly suitable for comparison. As a way of summarising the MSD the diffusion constant $D$ is measure of the rate of diffusion and the relationship
\begin{equation}
    D = \frac{1}{4}\ddiff{\,MSD(t)}{t}
\end{equation} 
can be derived from the principles of Brownian motion~\tocite. The diffusion constant is the slope of the MSD at the long time limit, when the MSD is obeying a $t^1$ power law.

\begin{figure}
    \centering
    \includegraphics[width=0.5\textwidth]{{{msd}}}
    \caption{The mean squared displacement measures the distance the centers of mass ($\times$) move between the initial (grey) and final (black) positions.}
    \label{fig:msd ex}
\end{figure}

The second dynamic quantity we are interested in is the rotational relaxation, a measure of the angular mobility of the molecule~\figref{rot ex}. The \emph{rotational relaxation} $C_n(t)$ is given by
\begin{equation}
    C_n(t) = \langle P_n[\vect{\hat{e}}(0) \cdot \vect{\hat{e}}(t)] \rangle
    \label{eq:rot}
\end{equation}
where $P_n$ is the $n$\textsuperscript{th} order Legendre polynomial, $\vect{\hat{e}}(t)$ is the orientation vector at time $t$, and the angle brackets $\langle\,\rangle$ denote the averaging over all molecules. We are primarily interested in the first two rotational relaxation functions $C_1$ and $C_2$, in which the Legendre polynomials take the form
\begin{align}
    P_1(x) &= x,
    P_2(x) &= 2x^2 - 1
\end{align}
as they have experimental analogues in dielectric relaxation, NMR, and optical methods~\cite{ediger:12}. The difference between the first order $C_1$ and second orer $C_2$ relaxation function is that the second order function is looking at smaller relaxations. The shape of both rotational relaxation functions is of exponential decay, meaning an appropriate representative value of these functions is the time it takes to reach $1/\e$. Using this definition we can define a rotational relaxation time $\tau_n$ such that the curve $\e^{-t/\tau_n}$ is a good representation of the behaviour of the relaxation. 

\begin{figure}
    \centering
    \includegraphics[width=0.5\textwidth]{{{rot}}}
    \caption{The rotational relaxation measures how much molecules have rotated (black) from their initial position (grey) taking the dot product of the orientation vectors of each molecule.}
    \label{fig:rot ex}
\end{figure}

In an real world system inelastic neutron scattering represents the most detailed account of relaxation in a liquid, combining information about the density fluctuations over a range of different length scales~\cite{perera:99}. The time correlation function associated with inelastic neutron scattering is the \emph{self-intermediate scattering function}, defined as
\begin{equation}
    F_{S}(k,t) = \left \langle\e^{i\vect{k} \cdot [\vect r(t)-\vect r(0)]}\right \rangle
\end{equation}
where $\vect k$ is the wave vector, the magnitude of which is the magnitude of the density fluctuations, $\vect r(t)$ is the position at time $t$, and the angle brakets $\langle\,\rangle$ denote the averaging over all molecules and an angular average over the direction of the wave vector $\vect k$. A computationally simpler function that measures these same density relaxations is the \emph{structural relaxation function} and is defined by
\begin{align}
    F_d(t) = \left \langle \begin{cases}
        \quad0 &\text{if}\quad \delta > d \\
        \quad1 &\text{if}\quad \delta \leq d
    \end{cases} \quad \right \rangle
    \label{eq:struct}
\end{align}
where $d$ is the magnitude of the fluctuation, $\delta$ is the fluctuation from the initial position~\eqref{msd}, and the angle brackets denote averaging over all molecules~\cite{widmer:12}.

For our molecular liquids we shall define an analogue of the structural relaxation function~\eqref{struct} defined in terms of each individual particle. Specifically we average over the number of particles, giving the number of particles that remain with the threshold distance $d$ at time $t$~\figref{stuct ex}. We will use a threshold value of \num{0.3}, this is small enough that both translational and rotational motion of a molecule contribute to the structural relaxation. As a way of a characteristic value of the structural relaxation we will be treating it in the same way as the rotational relaxation functions as the structural relaxation also obeys an exponential relaxation. We can define the structural relaxation time $\tau_s$ as the time for the structural relaxation function to reach a value of $1/\e$.

\begin{figure}
    \centering
    \includegraphics[width=0.5\textwidth]{{{struct}}}
    \caption{The structure function is a measure of how many particles have moved out of their initial positions (grey) with the threshold shown with a dotted green circle. Both rotational motion (depicted) and translational motion will result in particles moving out of this threshold region.}
    \label{fig:struct ex}
\end{figure} 

\section{Relaxation Dynamics of the Model Molecules; \done, \dcon and \tri}

In this Section we will be presenting the relaxation dynamics of each molecule over a range of temperatures. These temperature ranges reflect the temperature range over which the dynamics of each individual molecule were sufficiently large to allow relaxation. The simulations were monitored for signs of crystallisation at each temperature and those which showed crystallisation were discarded to ensure the relaxation dynamics measured are a property of the liquid phase.

The mean squared displacements~\figref{msd} of each molecule show a similar features. At short times (\numrange{e-3}{e-1}) the MSD is in the ballistic regime, this is where molecules are moving away from their initial position at constant velocity before they have collided with other molecules. The ballistic regime is characterised by a $t^2$ power law. At long times (\numrange{e5}{e7}) we have the diffusive regime, where the molecules are undergoing random motions. The diffusive regime is characterised by a $t^1$ power law. At high temperatures (3.50) we there is a direct transition from the ballistic regime to the diffusive regime, however at lower temperatures this transition is interrupted by a plateau region. The plateau region is where the molecules remain caged in their local environments and are unable to diffuse, as we decrease the temperature the plateau region extends in duration.

\begin{figure}
    \centering
    \begin{subfigure}[t]{\linewidth}
        \centering
        \includegraphics[width=0.6\textwidth]{{{Snowman-0.637556-1.0-msd}}}
        \caption{\done}
        \label{fig:done msd}
    \end{subfigure}
    \begin{subfigure}[t]{\linewidth}
        \centering
        \includegraphics[width=0.6\textwidth]{{{Snowman-0.637556-1.637556-msd}}}
        \caption{\dcon}
        \label{fig:dcon msd}
    \end{subfigure}
    \begin{subfigure}[t]{\linewidth}
        \centering
        \includegraphics[width=0.6\textwidth]{{{Trimer-0.637556-1.00-120-msd}}}
        \caption{\tri}
        \label{fig:tri msd}
    \end{subfigure}\hfill
    \caption{The Mean Squared Displacement, a measure a translational motion of the \done \subfigref{done msd}, \dcon \subfigref{dcon msd}, and \tri \subfigref{tri msd} molecules. The grey line indicates the $t^1$ power law defining the diffusive regime.}
    \label{fig:msd}
\end{figure}

The rotational relaxations~\figref{c1} of each molecule all show similar features. At high temperatures (\num{3.50}) we see exponential relaxation for all molecules. As we move to lower temperatures there is a deviation from this single exponential relaxation, we see a small initial relaxation to a plateau region, followed by an exponential relaxation. This two step relaxation process is similar to what we observe in the MSD, at low temperatures the relaxation is constrained by the local cage of molecules with relaxation occurring once escaped from the cage.

\begin{figure}
    \centering
    \begin{subfigure}[t]{\linewidth}
        \centering
        \includegraphics[width=0.6\textwidth]{{{Snowman-0.637556-1.0-C_1}}}
        \caption{\done}
        \label{fig:done c1}
    \end{subfigure}
    \begin{subfigure}[t]{\linewidth}
        \centering
        \includegraphics[width=0.6\textwidth]{{{Snowman-0.637556-1.637556-C_1}}}
        \caption{\dcon}
        \label{fig:dcon c1}
    \end{subfigure}
    \begin{subfigure}[t]{\linewidth}
        \centering
        \includegraphics[width=0.6\textwidth]{{{Trimer-0.637556-1.00-120-C_1}}}
        \caption{\tri}
        \label{fig:tri c1}
    \end{subfigure}\hfill
    \caption{The rotational relaxation function $C_1$, a measure of rotational motion for the \done \subfigref{done c1}, \dcon \subfigref{dcon c1}, and \tri \subfigref{tri c1} molecules.}
    \label{fig:c1}
\end{figure}

The structural relaxations~\figref{F}, have some interesting features. At short timescales (\numrange{e-3}{e-1}) the molecules have not had enough time to move the required distance for structural relaxation. At high temperatures we then see a sharp dropoff, characteristic of the step function used, to an exponential relaxation. However at low temperatures we begin to see a region with oscillations~\figref{tri F}, these oscillations are a result of entire system vibrations~\cite{perera:99} moving particles across the cutoff distance. This oscillatory region is similar to the plateau region in the rotational relaxation in that it is the transition region between the two steps of a two step relaxation process. In the structural relaxation function this second relaxation has the shape of an exponential function.

\begin{figure}
    \begin{subfigure}[t]{\linewidth}
        \centering
        \includegraphics[width=0.6\textwidth]{{{Snowman-0.637556-1.0-F}}}
        \caption{\done}
        \label{fig:done F}
    \end{subfigure}
    \begin{subfigure}[t]{\linewidth}
        \centering
        \includegraphics[width=0.6\textwidth]{{{Snowman-0.637556-1.637556-F}}}
        \caption{\dcon}
        \label{fig:dcon F}
    \end{subfigure}
    \begin{subfigure}[t]{\linewidth}
        \centering
        \includegraphics[width=0.6\textwidth]{{{Trimer-0.637556-1.00-120-F}}}
        \caption{\tri}
        \label{fig:tri F}
    \end{subfigure}\hfill
    \caption{The structural relaxation function $F$ for the \done \subfigref{done F}, \dcon \subfigref{dcon F}, and \tri \subfigref{tri F} molecules.}
    \label{fig:F}
\end{figure}

Comparing the dynamic quantities of the three molecules~\figref{dynamic comparison} using the characteristic values of the dynamic quantities there are large differences between the molecules. Looking at the diffusion constant~\figref{D}, \done shows a linear decrease in the diffusion constant as the temperature decreases. This linear behaviour is consistent with the Arrhenius relation and indicates that the molecular rearrangements that take place have a constant activation energy. The \dcon molecule shows definite non Arrhenius behaviour, with the diffusion constant diverging from that of \done by more than two orders of magnitude, while the \tri molecule tracks between the \done and \dcon molecules. Both the rotational relaxation~\figref{t1} and structural relaxation~\figref{ts} show similar behaviour to the diffusion constant, \done shows linear temperature dependence, the behaviour of a strong glass forming liquid. The \dcon molecule exhibits a significant increase as the temperature decreases, the behaviour of a fragile liquid. The final molecule, \tri sits between the \done and the \dcon molecules covering the range of dynamic behaviour of liquids.

The structural relaxation~\figref{ts} displays two gradients for each of the molecules, indicative of two separate relaxation processes with different activation energies. At high temperatures there is a process with a low activation energy which suddenly transitions at low temperatures to a process with a high activation energy. Neither the diffusion constant or the rotational relaxation show this sudden transition between two processes. By multiplying the diffusion constant~$D$ by the structural relaxation time~$\tau_s$ \figref{D.ts} the change in relaxation process is clearly seen as the inflection point. Initially the structural relaxation is dominated by translational motion, a process with a low activation energy. However at the inflection point the process changes to one dominated by rotational motion with at higher activation energy. Further evidence of this transition is comparing the activation energies of the rotational~\figref{t1} and structural~\figref{ts} relaxations, after the transition of the structural relaxation to a rotationally dominated process the activation energies show strong correlations. Despite the sudden change in the structural relaxation time, there is no corresponding discontinuity in the diffusion constant~\figref{D}. In defining the diffusion constant we are only interested in the gradient of the MSD in the diffusive regime, ignoring the time to reach the diffusive regime. At low temperatures the plateau region of the MSD is sits below the 0.09 cutoff of the structure function, during which rotations are the primary contributor to the structural relaxation. 

\begin{figure}
    \centering
    \begin{subfigure}{\linewidth}
        \centering
        \includegraphics[width=0.6\textwidth]{D}
        \caption{}
        \label{fig:D}
    \end{subfigure}
    \begin{subfigure}{\linewidth}
        \centering
        \includegraphics[width=0.6\textwidth]{t1}
        \caption{}
        \label{fig:t1}
    \end{subfigure}
    \begin{subfigure}{\textwidth}
        \centering
        \includegraphics[width=0.6\textwidth]{ts}
        \caption{}
        \label{fig:ts}
    \end{subfigure}
    \caption{Comparison of the diffusion constant~\subfigref{D}, rotational relaxation time $\tau_1$~\subfigref{t1}, and structural relaxation time $\tau_s$~\subfigref{ts} for the three molecules that we are studying. As we move from \done to \tri to \dcon the dynamics of the liquid become increasingly non-Arrhenius.}
    \label{fig:dynamic comparison}
\end{figure}

\section{Analysis of Coupling of Diffusion and Rotation}

\begin{figure}
    \centering
    \includegraphics[width=0.6\textwidth]{{{D.ts}}}
    \caption{Relative contribution of diffusion to structural relaxation over a number of temperatures. The discontinuity shows the point at which the structural relaxation moves from being dominated by translational motion to being dominated by rotational motion.}
    \label{fig:D.ts}
\end{figure}

The increased contribution of the rotations to the overall dynamics is an interesting result, previous studies both experimental~\cite{stillinger:94,cicerone:95,debenedetti:01,swallen:03} and simulated~\cite{kammerer:97} have found that translations, rather than rotations dominate at low temperatures contrary to our observations of structural relaxations. Further support for the dominance of rotations in our systems is seen by taking the product of the diffusion constant and the rotational relaxation~\figref{D.t1}, the diffusion constant is getting smaller much faster than the increase in the rotational relaxation time. One possible explanation for the dominance of rotations over translations at low temperature is that when molecules interlock via a concavity the concavity inhibits the translational motion, molecules have to unlock from the concavity to move past one another. Rotations however, are possible while still retaining the interaction at the concavity.

\begin{figure}
    \centering
    \includegraphics[width=0.6\textwidth]{{{D.t1}}}
    \caption{Relative contributions of the diffusion constant and rotational relaxation to the overall dynamics. The decrease as the temperature decreased indicates the translational diffusion is getting slower faster than the rotational diffusion.}
    \label{fig:D.t1}
\end{figure}

Studying the rotations of our molecules further, results found in simulations of similar molecules~\cite{kammerer:97,michele:01} found that at low temperatures molecules underwent rotational relaxation by flips of \ang{180}. To investigate the contribution of \ang{180} jumps in our molecules we can compare the first and second order rotational relaxation times since flips of \ang{180} do not contribute to the second order relaxation. Taking the ratio of the first and second relaxation times $\tau_1/\tau_2$ we can determine whether these \ang{180} are present in our simulations~\figref{t1/t2}. We see a decrease in this ratio for all molecules, $\tau_2$ is larger than $\tau_1$ at low temperature meaning there is little contribution to the relaxation be \ang{180} flips in orientation.

\begin{figure}
    \centering
    \includegraphics[width=0.6\textwidth]{{{t1.t2}}}
    \caption{The ratio of the first and second order rotational relaxation times as a function of temperature. As the temperature decreases the decrease in this ratio shows that molecules undergo small angle changes rather than \ang{180} flips.}
    \label{fig:t1/t2}
\end{figure}

In describing the differences between the dynamic properties of the molecules the Aperiodic Crystal Structure model provides a relationship between the rotational relaxation and the diffusion constant and the local structural rigidity in the Hall-Wolynes equation~\cite{hall:87,dyre:96}
\begin{equation}
    \tau_1, D \propto \e^{a^2/2\langle u^2\rangle}
\end{equation}
where $a$ is the displacement to overcome the local energy barrier and $\langle u^2 \rangle$ is the Debye-Waller factor, a measure of the local rigidity of the structure. The Debye-Waller factor is found as the MSD at the time where 
\begin{equation}
    \ddiff{\log{(\text{MSD})}}{\log{(t)}}
\end{equation}
is a minimised~\cite{larini:08}, the value where the slope is lowest in \textfigref{snowman 0.637556 1.0 msd}.
\towrite{DW}\figref{DW}.

Plotting the diffusion constant~\figref{DW.D} and the rotational relaxation time~\figref{DW.t1} against $1/\langle u^2 \rangle$ we see that instead of the molecules diverging there is a much better correlation between them. The Debye-Waller factor normalises the dynamics based on the local structural rigidity explaining a significant portion of the difference in the dynamics of the molecules. In changing the shape of molecules we change the local stiffness of the liquid phase resulting in order of magnitude changes of the dynamics. The cause of the temperature dependence of the local stiffness is likely a key contributor to the fragility of a liquid.

\begin{figure}
    \centering
    \includegraphics[width=0.6\textwidth]{DW}
    \caption{The temperature dependence of the Debye-Waller factor for each molecule}
    \label{DW}
\end{figure}

\begin{figure}
    \begin{subfigure}{\textwidth}
        \centering
        \includegraphics[width=0.6\textwidth]{{{DW.D}}}
        \caption{}
        \label{fig:DW.D}
    \end{subfigure}
    \begin{subfigure}{\textwidth}
        \centering
        \includegraphics[width=0.6\textwidth]{{{DW.t1}}}
        \caption{}
        \label{fig:DW.t1}
    \end{subfigure}
    \caption{By normalising the diffusion constant~\subfigref{DW.D} and rotational relaxation~\subfigref{DW.t1} by the Debye-Waller factor we are able to account for significant differences in the dynamic quantities. At high temperatures the Debye-Waller factor $\langle u^2 \rangle$ is a fairly constant large value (left of figure). As the temperature decreases both \done and \tri show similar rotational relaxation times. The \dcon molecule shows a steeper slope indicating the presence of more complicated process.}
\end{figure}

\section{Summary}

In this Chapter we have found that small changes to molecular shape play an important role in the dynamics of the liquid phase, resulting in relaxation times that differ by up to four orders of magnitude. Much of these differences in dynamics can be explained by the local structural rigidity. It is structure and how it is influenced by the shape of the molecules that we want to investigate in the following Chapter.

