\chapter{Liquid Dynamics}

Molecules (<shape>-<radius>-<distance>-<theta>)
\begin{itemize}
    \item Snowman-0.637556-1.0
    \item Snowman-0.637556-1.637556
    \item Trimer-0.637556-1.0-120
\end{itemize}


Data
\begin{itemize}
    \item Snowman-0.637556-1.0
        \begin{itemize}
            \item T = 
        \end{itemize}
    \item Snowman-0.637556-1.637556
        \begin{itemize}
            \item T = 
        \end{itemize}
    \item Trimer-0.637556-1.0-120
        \begin{itemize}
            \item T = 
        \end{itemize}
\end{itemize}


Analysis
\begin{itemize}
    \item MSD
    \item Diffusion Constant
    \item Rotational relaxation (R1, R2)
    \item Rotational relaxation time (t1, t2)
    \item Structure function (f)
    \item Structural relaxation time (ts)
    \item Order parameter - neighbouring small and large particles
    \item Hexatic ordering
\end{itemize}

Key Plots
\begin{itemize}
    \item MSD for all T
    \item Rotations for all T
    \item Relaxation times function of 1/T
    \item Comparing d=1 to d=1+r
        \begin{itemize}
            \item Diffusion constant
            \item Rotational relaxations
            \item Structure functions
        \end{itemize}
\end{itemize}


\section{Characterisation}

The dynamics of a molecular system is characterised by two main elements, the diffusion of the particles through the liquid and the rotational motion of those particles. Understanding these properties of the liquid provide the basis for further study giving estimates of the timescales that events are likely to occur at. It also provides important information about the range of temperatures at which the particular system can be studied. The molecules that we have chosen for study are shown in \textfigref{shapes}, these molecules were chosen to be representative of the range of samples and also with the possibility of showing interesting behaviour as discussed in \textsecref{shapes of interest}.

\begin{figure}
    \todofigure{Shapes being studied for dynamics}
    \caption{The shapes being studied in this chapter}
\end{figure}



 \section{d=1+r}

\section{Oh hey they are different}
