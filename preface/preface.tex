
\beforepreface

\prefacesection{Abstract}

Why do we care about the problem and the results:


Compared to metals, molecules are phenomenally bad at crystallising, with no understanding as to why this is the case. One property characteristic of a molecule is its shape. We used molecular dynamics simulations to study the role of molecular shape on the dynamic and structural properties of condensed phases. Small changes to the shape of a molecule resulted in order of magnitude changes in the dynamic properties, while the structural properties were markedly different. Despite the large changes in dynamics, when normalised using the Debye-Waller factor the dynamics were correlated. It was also found that the dynamics of molecular liquids are incredibly slow at the melting point, such that crystallisation takes place slowly. The shape of a molecule is an important contributing factor to the dynamics, playing a role in the rigidity of the local structures and the slow formation of crystalline order.


It was also found that dynamics at the freezing point are incredibly slow, suggesting that the liquid phase is stabilised by the additional entropy from the rotational orientation. Shape plays an important role in the properties of a 


Molecular glasses display a wide variety of structural


While we have a good understanding of the liquid and crystal phases, however we do not fully understand the formation of solid amorphous phases. Much of the 

In this 


What probelm are you trying to solve:

Understanding the role of molecular shape in the dynamic and structural properties of the condensed phases of those molecules.

How did you solve the problem:

Using molecular dynamics

What is the answer:

Small changes in shape can lead to orders of difference in the dynamics of a system.
The additional rotational entropy of a molecule stabilises the liquid phase reducing the melting point to a temperature at which the dynamics are incredibly slow.

Conclusions:

Our understanding of glass formining liquids in simulations is based on binary mixtures. Real world molecular glass formers are single component systems. This gives us a way to understnand.



\towrite{Abstract}

\prefacesection{Statement of Contribution}

\towrite{Contributions}

\contributionsignature

\prefacesection{Acknowledgements}

\towrite{Acknowledgements}

\prefacesection{Publications}

Work from Chapter 3 has been submitted for publication

\towrite{publication details}

\todototoc\listoftodos

\afterpreface

