\chapter{Packing shapes on a Plane}

Shapes studying:
\begin{itemize}
    \item Shape - Snowman, Trimer
    \item Radius - 0.5 0.6 0.7 0.8 0.9 1.0
    \item Distance - 1.0 1+r
    \item Theta - 120, 180
\end{itemize}

Data gathered for each shape
\begin{itemize}
    \item High temp quench - T = 5.0
    \item Low temp quench - T = 0.5
    \item Optimal packing, structure from Toby, perform MD equil
\end{itemize}

Analysis
\begin{itemize}
    \item Configuration image
    \item Contact numbers ($f_6$)
    \item Radial distribution function ($g(r)$)
    \item Structure function ($s(q)$)
    \item Hexatic ordering 
\end{itemize}

Important figures
\begin{itemize}
    \item $f_6$, $s(q)$, hexatic function of radius
    \item Packing fraction comparison
    \item Comparison of high and low temp quench
\end{itemize}

\section{Packing Dimers}

It is widely documented that shape plays an important role in crystal packing, with small changes to the shape dramatically affecting the crystal structure. Amorphous packing however is an unstructured arrangement, small changes would just get averaged over the system.


I have studied how a number of small modifications to the shape of a molecule affects the amorphous packings. The amorphous packings are inherent structures taken from a liquid or glassy phase. The first of these is the effect of the relative sizes of the two discs. There are a number of binary disc mixtures that are glass forming~\tocite, the two sizes of discs incapable of packing better than individually. There are also a number of binary disc packings that provide optimal space coverage~\tocite, the small particles perfectly filling the gaps left by the large particles. When the two discs ($r_1 = 1, r_2 = r$) are held fixed at a distance of $1+r$ the packing is conceptually similar to a binary disc mixture however there is the extra constraint that the distance and orientation remains fixed. \textfigref{fix d var r} shows how the packing fraction compares to the optimal packing fraction, the optimal packings were found by Toby Hudson~\tocite and equilibrated using the same potential as the amorphous systems\toref. At $r = 1$ both the amorphous and the optimal crystal packing have the same packing fraction, looking at their configurations~\figref{config:d2r1} the individual particles pack as though they are just discs, the packing efficiency of individual particles is the same, however the rotational orientation of the molecules in the crystal structure is ordered, while the amorphous structure has the molecules randomly arranged in the 6 possible orientations\tocheck. This means the packing is not strictly amorphous.

\begin{figure}
    \todofigure{r = 0.5 - 1.0, d = 1+r vs packing fraction, compare to optimal}
    \caption{The packing efficiencies of amorphous snowmen compared to their optimal packing fraction.}
    \label{fig:fix d var r}
\end{figure}

\begin{figure}
    \todofigure{configuration of d=1 r=2 for both amorphous and crystal showing alignment of the molecules}
    \label{fig:config:d1r2}
    \caption{Both the amorphous and the crystal structure pack in the same way with only the orientations of particles differing between them}
\end{figure}

Another special crystal structure is where $d=1.637556$, this is one of the optimal binary packings~\figref{crys:d=1.637556}. Unlike where $d=2$ the packing fraction of this amorphous configuration is far lower than the crystal structure. This is interesting as both crystal structures are from the p2 or p2gg space groups, however the $d=2$ structure is easier to form than the $d=1.637556$.

\begin{figure}
    \todofigure{Structure of d=1.637556 snowman}
    \caption{Optimal packing of binary discs, small disc is 0.637556 times the size of the large disc.}
    \label{crys:d=1.637556}
\end{figure}

At smaller radii 
\towrite{what the smaller radii do, how they compare to the crystal packing}

The case where the two particles are at a distance of $1+r$ is an extreme case, molecules tend to have overlapping spheres, the overlap responsible for the bonding between them. As a result of this it makes sense to investigate the effect of distance on the amorphous packing. In \textfigref{var d fix r} shows how the packing fraction changes as a function of the distance where the radius of both particles is the same. There is a significant increase in the packing fraction as the radius decreases, this is as a result of the overlap of the particles filling more space. Despite this increase in the packing fraction the packing becomes amorphous as the distance decreases, then increasing again as the distance goes to 0. Rather than looking at every configuration and we need an order parameter to quantify the amount of order within the system.

\begin{figure}
    \todofigure{plot of d vs packing fraction for r = 1}
    \label{fig:var d fix r}
    \caption{Some words}
\end{figure}

\section{Order parameters}

There are many order parameters suitable for different types of systems, some more suitable than others.
\begin{itemize}
    \item Circle order (number of neighbours for d=1.637556)
    \item Packing fraction
    \item structure factor
    \item 6 fold order
    \item g(r)
    \item short range order
    \item orientational order
\end{itemize}

\begin{itemize}
    \item The 6 fold order parameter is most suitable for degree of crystallinity.
    \item g(r) for d=1 snowman

\towrite{fill out $\uparrow$}

\begin{figure}
    \begin{subfigure}{0.5\textwidth}
        \todofigure{Various order parameters for crystal and amorphous as a function of r}
        \caption{Order parameters for radii}
    \end{subfigure}
    \begin{subfigure}{0.5\textwidth}
        \todofigure{Various order parameters for crystal and amorphous as a function of d}
    \end{subfigure}
\end{figure}

\section{Packing Trimers}
\label{packing trimers}

There is another simple property that has the potential to have an effect on the amorphous structures, angle. To study angle we need to move from two particles to three.

\section{Shapes of Interest}
\label{shapes of interest}

The range of shapes, even simple ones is huge with a wide range of behaviour. It is not feasible to study this entire range of shapes in detail, the list of shapes need to be narrowed down to a range that is both interesting and representative of the wider spectrum of shapes.

\towrite{I chose d=1.637556 snowmen, here's why}

